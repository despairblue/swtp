% !TEX encoding = UTF-8 Unicode
% !TEX TS-program = Make

\documentclass[a4paper, 10pt]{article}
\usepackage[utf8]{inputenc} 
\usepackage[ngerman]{babel} 
\usepackage{coordsys,logsys,color} 
\usepackage{fancyhdr} 
\usepackage{hyperref}
\usepackage{texdraw}
\usepackage{colortbl}
\usepackage[xindy, acronym, toc, style=altlist]{glossaries}

\input{txdtools}

\NeedsTeXFormat{LaTeX2e}
\ProvidesPackage{hyperref}
\definecolor{darkblue}{rgb}{0,0,.6}
\hypersetup{pdftex=false, colorlinks=true, breaklinks=true, linkcolor=darkblue, menucolor=darkblue, pagecolor=darkblue, urlcolor=darkblue, citecolor=darkblue}

\pagestyle{fancy}

\makeglossary
% !TEX encoding = UTF-8 Unicode
% !TEX root = Pflichtenheft.tex

\newacronym{GUI}{GUI}{Graphical User Interface}
\newacronym{CSV}{CSV}{\gls{glos:CSV} \protect\glsadd{glos:CSV}}

\newglossaryentry{Canvas} {
	name={Canvas},
	description={}
}

\newglossaryentry{Drag and Drop} {
	name={Drag and Drop},
	description={}
}

\newglossaryentry{glos:CSV} {
	name={Comma-Separated Values},
	description={Das Dateiformat CSV beschreibt den Aufbau einer Textdatei zur Speicherung oder zum Austausch einfach strukturierter Daten}
}

% !TEX encoding = UTF-8 Unicode
% !TEX root =  Pflichtenheft.tex

\renewcommand{\familydefault}{cmss}

\definecolor{fgcgray}{rgb}{0.4, 0.4, 0.4}
\definecolor{warning}{rgb}{0.9, 0.1, 0.0}
\definecolor{bgctitle}{rgb}{0.5, 0.5, 0.5}
\definecolor{fgctitle}{rgb}{0.95, 0.95, 0.95}

\definecolor{Gray}{gray}{0.8}
\definecolor{lightGray}{gray}{0.925}

\newcolumntype{A}{>{\columncolor{Gray}}c}
\newcolumntype{B}{>{\columncolor{Gray}}l}

\newcommand{\titlefont}[1]{\textcolor{fgctitle}{\fontfamily{cmss}\fontseries{bx}\fontshape{n}\fontsize{20.48}{0pt} \selectfont #1}}
\newcommand{\inversetitlefont}[1]{\textcolor{bgctitle}{\fontfamily{cmss}\fontseries{bx}\fontshape{n}\fontsize{20.48}{0pt} \selectfont #1}}

\addtolength{\oddsidemargin}{-1.0cm}
\addtolength{\evensidemargin}{-1.0cm}
\addtolength{\headwidth}{2.0cm}
\addtolength{\textwidth}{2.0cm}

\setlength{\parindent}{0cm}

\renewcommand{\labelitemi}{$\circ$}
\renewcommand{\labelitemii}{$\diamond$}

\newcommand{\spaceline}[1][8pt]{\vskip #1}
\newcommand{\comment}[1]{\spaceline[5pt] \textcolor{fgcgray}{\scriptsize #1} \spaceline[15pt]}
\newcommand{\attrname}[1]{\textcolor{fgcgray}{\scriptsize #1}}


\makeatletter

\newcommand*{\project}[1]{\gdef\@project{#1}}
\newcommand*{\projectnumber}[1]{\gdef\@projectnumber{#1}}

\newcommand*{\version}[1]{\gdef\@version{#1}}
\newcommand*{\home}[1]{\gdef\@home{#1}}
\newcommand*{\homeref}[1]{\gdef\@homeref{#1}}
\newcommand*{\prerequisite}[1]{\gdef\@prerequisite{#1}}
\newcommand*{\prerequisiteref}[1]{\gdef\@prerequisiteref{#1}}

\newcommand*{\firststudent}[4]{
	\gdef\@firststudentname{#1}
	\gdef\@firststudentmatrnr{#2}
	\gdef\@firststudentstudypath{#3}
	\gdef\@firststudentemail{#4}
}

\newcommand*{\secondstudent}[4]{
	\gdef\@secondstudentname{#1}
	\gdef\@secondstudentmatrnr{#2}
	\gdef\@secondstudentstudypath{#3}
	\gdef\@secondstudentemail{#4}
}

\def\@maketitle{
  %\begin{titlepage}
  \begin{center}
    \colorbox{bgctitle}{
      \parbox{\textwidth}{
        \spaceline
        \centering{\titlefont{\@title}}
        \par
        \spaceline
      }
    }
    \colorbox{white}{
      \parbox{\textwidth}{
        \spaceline
        \centering{\inversetitlefont{\@project}}
        \par
        \spaceline
      }
    }
  \end{center}
  \spaceline[1.5em] {
    \begin{flushright}
    \begin{tabular}[t]{rl}
      \attrname{Projekt:} & \@project ~ \@version \\
      \attrname{Voraussetzung:} & \href{\@prerequisiteref}{\@prerequisite} \\
      \attrname{Autor:} & \@author \\
      \attrname{Home:} & \href{\@homeref}{\@home} \\
      \attrname{letzte "Anderung:} & \@date
    \end{tabular}
    \end{flushright}
    \par
  }
  \spaceline[5.5em]
  %\end{titlepage}
}

\newcommand*{\makesteinbach} {
	\large{
		\begin{center}
			\begin{tabular}{|A|}
				\hline
				\\
				\textbf{\Huge{\quad \quad \quad \quad \quad Pflichtenheft \quad \quad \quad \quad}}
				\\
				\\
				zum Softwareprojekt\\
				(Prof. Steinbach)\\
				\\
				\hline
			\end{tabular}
		\end{center}
	
		\vspace{1mm}
	
		\begin{center}
			\begin{tabular}{|A|}
				\hline
				\\
				\textbf{\quad \quad \quad \@project ~ (\@projectnumber) \quad \quad \quad}
				\\
				\\
				\hline
			\end{tabular}
		\end{center}
		
		\vspace{15mm}
		
		\textbf{Angaben zu den am Projekt beteiligten Studenten:}
		
		\begin{center}
			\normalsize{
				\begin{tabular}{|c|l|l|l|l|}
					\hline
					\rowcolor{Gray} & \textbf{Name, Vorname} & \textbf{Mat.-Nr.} & \textbf{Studiengang} & \textbf{Email-Adresse} \\
					\hline
					\rowcolor{lightGray} \textbf{1.} &  \@firststudentname &  \@firststudentmatrnr &  \@firststudentstudypath &  \@firststudentemail \\
					\hline
					\rowcolor{Gray} \textbf{2}. &  \@secondstudentname&  \@secondstudentmatrnr &  \@secondstudentstudypath &  \@secondstudentemail \\
					\hline
					\rowcolor{lightGray} \textbf{3.} & & & & \\
					\hline
					\rowcolor{Gray} \textbf{4.} & & & & \\
					\hline
					\rowcolor{lightGray} \textbf{5.} & & & & \\
					\hline
				\end{tabular}
			}
		\end{center}
		
		\vfill
		
		\begin{center}
			\begin{tabular}{|BB|}
				\hline
				\textbf{Bestätigt durch Prof. Steinbach} & \quad \quad \quad \quad \quad \quad \quad \quad \quad \\
				\textbf{Datum, Unterschrift} &
				\\
				\hline
			\end{tabular}
		\end{center}
	}
}


\setcounter{secnumdepth}{4}
\setcounter{tocdepth}{4}
	 
\newcounter{subsubsubsection}[subsubsection]
\def\subsubsubsectionmark#1{}
\def\thesubsubsubsection{\thesubsubsection .\arabic{subsubsubsection}}
\def\subsubsubsection{\@startsection{subsubsubsection}{4}{\z@}{-3.25ex plus -1 ex minus -.2ex}{1.5ex plus .2ex}{\normalsize\bf}}
\def\l@subsubsubsection{\@dottedtocline{4}{4.8em}{4.2em}}

\makeatother

\everytexdraw{
  \drawdim cm \linewd 0.01
  \arrowheadtype t:T
  \arrowheadsize l:0.2 w:0.2
  \setgray 0.5
}

\newcommand{\xheight}{0.6}
\newcommand{\xlength}{0.6}
\newcommand{\yheighta}{1.0}
\newcommand{\yheightb}{0.8}
\newcommand{\yheightc}{0.6}
\newcommand{\yheightd}{0.5}
\newcommand{\yheighte}{0.4}
\newcommand{\yheightf}{0.35}

\newcommand{\xhline}{\rlvec({\xlength} 0)}
\newcommand{\xharrow}{\ravec(0.7 0)}
\newcommand{\xnext}{%\rlvec(0.05 0) \lpatt(0.04 0.04) \rlvec(0.15 0) \lpatt()
}

\newcommand{\xtext}[3][\xheight]{
  \bsegment
    \bsegment
      \setsegscale 0.5
      \textref h:L v:C  \htext({\xheight} -0.1){#3}
    \esegment
    \setsegscale 0.5 \lvec(0 #1)
    \setsegscale 1
    \rlvec(#2 0) \rlvec(0 -#1) \rlvec(-#2 0) \lvec(0 0)
    \savepos(#2 0)(*@x *@y)
  \esegment
  \move(*@x *@y)
}

\newcommand{\bxtext}[3][\xheight]{
  \setgray{0.1}
  \linewd{0.026}
  \xtext[\xheight]{#2}{#3}
  \linewd{0.01}
  \setgray{0.5}
}

\newcommand{\xstartpage}{\bxtext{2.1}{Startseite}}
\newcommand{\xmainpage}{\bxtext{2.3}{Hauptseite}}
\newcommand{\xusermenu}{\bxtext{2.9}{Benutzermenu}}
\newcommand{\xgamelist}{\bxtext{2.2}{Spieleliste}}
\newcommand{\xportfolio}{\bxtext{2.0}{Portfolio}}
\newcommand{\xaccount}{\xtext{3.8}{Kennung per \textsl{eMail}}}


\begin{document}
  % !TEX encoding = UTF-8 Unicode
% !TEX root =  Pflichtenheft.tex

\lhead{\sc{Lastenheft Lgk Smltr}}
%\cfoot{-~\thepage~-}

\title{Beispiel: Pflichtenheft}
\project{Logik Simulator}
\projectnumber{30} 
\firststudent{Robert Schneider}{52588}{BAI}{rob.schneider@student.tu-freiberg.de}
\secondstudent{Danny Arnold}{52315}{BAI}{danny.arnold@student.tu-freiberg.de}

%\maketitle

\makesteinbach

%\thispagestyle{empty}
%~
%\newpage
%\tableofcontents
 \newpage
  \tableofcontents \newpage
  % !TEX encoding = UTF-8 Unicode
% !TEX root =  Pflichtenheft.tex

\section{Zielbestimmungen}

%\comment{Welche Musskriterien, Wunschkriterien, Abgrenzungskriterien sind erforderlich?}

Es ist ein Programm gefordert, welches es einer Person ermöglicht Boolesche Schaltungen grafisch zu erstellen und zu simulieren.

\subsection{Musskriterien}

\begin{itemize}
	\item Die \gls{GUI}
	
	\begin{itemize}
		\item Kombinatorische Schaltungen erstellen können
		\item Darstellung eines \gls{Canvas} auf dem Gatter per \gls{Drag and Drop} angeordnet werden können
		\item Die Gatter sollen mit Leitungen verbunden werden können
		\item Eingangs- und Ausgangsgrößen sollen als Gatter auf dem gls{Canvas} angeordnet werden können.
		\item Leitungen sollen sich auch Aufspalten können, die Verbindung kann als Gatter realisiert werden.
		

	\end{itemize}

	\item Simulation
	
	\begin{itemize}
		\item Für die Eingänge soll es möglich sein eine Liste von Kombination von Eingangsgrößen angeben zu können (Einganggrößentabelle)
		\item Es soll möglich sein die Eingangsgrößentabelle Kombination für Kombination durchzugehen
		\item Es soll möglich sein die Eingangsgrößentabelle auf einmal abarbeiten zu lassen und sich für jede Zeile die Ausgangsgrößen in einer Tabelle in Verbindung mit der jeweiligen Kombination der Eingangsgrößen ansehen zu können
		\item Ob an einer Leitung ein Strom anliegt oder nicht soll farblich gekennzeichnet
		\item Der Signalverlauf ausgewählter Leitungen soll als Grafik angezeigt werden können
	\end{itemize}

	\item Sonstiges
	
	\begin{itemize}
		\item Das Speichern und Laden von Schaltungen soll möglich sein
		\item Die Eingangsgrößentabelle soll separat von der Schaltung abgespeichert und geladen werden können
	\end{itemize}
	
\end{itemize}

\subsection{Wunschkriterien}

\begin{itemize}
	\item Möglichkeit Sequentielle Schaltungen erstellen und simulieren zu können
\end{itemize}

\subsection{Abgrenzungskriterien}

\begin{itemize}
	\item keine
\end{itemize}
 \newpage
  % !TEX encoding = UTF-8 Unicode
% !TEX root =  Pflichtenheft.tex

\section{Produkteinsatz}

%\comment{Welche Anwendungsbereiche (Zweck), Zielgruppen (Wer mit welchen Qualifikationen), Betriebsbedingungen (Betriebszeit, Aufsicht)?}

\subsection{Anwendungsbereiche}

Das Produkt kann von jedem verwendet werden der Boolesche Schaltungen erstellen möchte, sei es für schulische Zwecke oder anderweitig.

\subsection{Zielgruppen}

Menschen die ein Interesse an Booleschen Schaltungen haben.\\

Es werden Grundkenntnisse im Umgang mit Computern vorausgesetzt. Es ist notwendig zu wissen wie man ein Programm startet und Text über die Tastatur eingibt und eine Maus benutzt.

\subsection{Betriebsbedingungen}

Das Program muss auf einem Windows-Computer mit .NET-Laufzeitumgebung laufen \newpage
  % !TEX encoding = UTF-8 Unicode
% !TEX root =  Pflichtenheft.tex

\section{Produktumgebung}

%\comment{Welche Software, Hardware und Orgware wird benötigt?}

\subsection{Software}

Zur Ausführung des Programms und des Clients muss auf dem Computer ein Windows mit .NET laufen.

\subsection{Hardware}

\begin{itemize}
	\item Pentium 2-GHz-Prozessor oder schneller
	\item Mindestens 512 MB RAM
	\item Mindestens 1,5 GB Speicherplatz auf der Festplatte
	\item Tastatur und Microsoft Maus oder ein kompatibles Zeigegerät
	\item Videoadapter und Monitor mit Super VGA (800 x 600) oder höherer Auflösung
\end{itemize}

\subsection{Orgware}

Keine. \newpage
  % !TEX encoding = UTF-8 Unicode
% !TEX root =  Pflichtenheft.tex

\section{Produktfunktionen}

%\comment{Was leistet das Produkt aus Benutzersicht?}

\subsection{Schaltplanerstellung}

Der Benutzer soll die Möglichkeit haben einen Schaltplan zu erstellen.
\begin{description}
	\item[/F010/] Möglichkeit Gatter von einer Toolbar per Drag and Drop in einem bestimmten Bereich des Fensters anzuordnen.
	\begin{description}
		\item[/F020/] zu implementierende Gatterarten: AND, OR, NOT, XOR, NOR, NAND
		\item[/F030/] besonderes Gatter: Knoten\\
		Wird benutzt um Leitungen zu teilen, damit ein Ausgangssignal zu mehreren Eingängen fließen kann
		\item[/WF040/] zu implementierende Gatterarten: FLIPFLOP, LATCH\\
		nötig für sequentielle Schaltungen
	\end{description}
	\item[/F050/] Möglichkeit die Gatter mittels Leitungen miteinander zu verbinden
	\item[/F060/] Möglichkeit Eingangsgrößen festzulegen
	\begin{description}
		\item[/F070/] Eingangsgrößen sollen separat auf dem \gls{Canvas} angeordnet werden können, ähnlich wie Gatter
	\end{description}
	\item[/F080/] Schaltungsausgänge festlegen
	\begin{description}
		\item[/F090/] Schaltungsausgänge sollen separat auf dem \gls{Canvas} angeordnet werden können, ähnlich wie Gatter
	\end{description}
\end{description}

\subsection{Simulation}

Der Benutzer soll den von ihm erstellten Schaltplan simulieren können. 
\begin{description}
	\item[/F100/] Für die Eingänge soll es möglich sein eine Liste von Kombination von Eingangsgrößen angeben zu können (Einganggrößentabelle)
	\item[/F110/] Es soll möglich sein die Eingangsgrößentabelle Kombination für Kombination durchzugehen
	\item[/F120/] Es soll möglich sein die Eingangsgrößentabelle auf einmal abarbeiten zu lassen und sich für jede Zeile die Ausgangsgrößen in einer Tabelle in Verbindung mit der jeweiligen Kombination der Eingangsgrößen ansehen zu können
	\item[/F130/] Signalverlauf in den Leitungen grafisch darstellen, Leitungen auf den eine Spannung anliegt werden rot gefärbt
	\item[/F140/] Signalverlauf von ausgewählten Leitungen soll als Grafik angezeigt werden
	
	\item[/WF150/] Möglichkeit Takt für Takt durchzugehen für sequentielle Schaltungen
	\item[/WF160/] Möglichkeit alle Zwischenergebnisse nach x Takten tabellarisch sich auszugeben lassen
\end{description}

\subsection{Sonstiges}

Der Benutzer kann Schaltungen speichern und laden.
\begin{description}
	\item[/F200/] Das Speichern und Laden von Schaltungen soll möglich sein
	\item[/F210/] Die Eingangsgrößentabelle soll separat von der Schaltung abgespeichert und geladen werden können
\end{description}
 \newpage
  % !TEX encoding = UTF-8 Unicode
% !TEX root =  Pflichtenheft.tex

\section{Produktdaten}

%\comment{Was speichert das Produkt (langfristig) aus Benutzersicht?}

Es sollen folgende Daten persistent gespeichert werden. \\
Die Speicherung erfolgt in einer simplen Textdatei (\gls{CSV}).\\
Eingangsgrößentabellen werden in separaten Texdateien gespeichert.
\begin{description}
	\item[D010/] \emph{Gatterdaten:} Alle Informationen zu einem Gatter.
	\begin{itemize}
		\item \emph{Typ}
		\item \emph{Name} 
		\item \emph{Position} 
	\end{itemize}
	
	\item[/D020/] \emph{Leitungsdaten:} Alle Informationen zu einer Leitung
	\begin{itemize}
		\item \emph{Name}
		\item \emph{Gatter A:} Das Gatter mit dessen Eingang die Leitung verbunden wird
		\item \emph{Gatter B:} Dast Gatter mit dessem Ausgang die Leitung verbunden wird
	\end{itemize}
	
	\item[/D030/] \emph{Eingangsgrößendaten:} Alle Information zu den Eingangsgrößen
	\begin{itemize}
		\item \emph{Name}
		\item \emph{Position}
		\item \emph{1 oder 0}
	\end{itemize}
	
	\item[/D040/] \emph{Ausgangsgrößen:} Alle Informationen zu den Ausgangsgrößen
	\begin{itemize}
		\item \emph{Name}
		\item \emph{Position}
	\end{itemize}
	
	\item[/D050/] \emph{Eingangsgrößentabelle:} Alle Informationen zu einer Eingangsgrößentabelle
	\begin{itemize}
		\item \emph{Anzahl der Eingänge}
		\item \emph{Die verschiedenen Belegungen der Eingänge}
	\end{itemize}
\end{description} \newpage
  % !TEX encoding = UTF-8 Unicode
% !TEX root =  Pflichtenheft.tex

\section{Produktleistungen}

%\comment{Welche zeit- und umfangsbezogenen Anforderungen gibt es?}

\begin{description}
	\item[/L010/] Mindestens 5 Eingänge
	\item[/L020/] Mindestens 5 Ausgänge
	\item[/L030/] Mindestens 50 Gatter
\end{description}

 \newpage
  % !TEX encoding = UTF-8 Unicode
% !TEX root =  Pflichtenheft.tex

\section{Benutzungsoberfläche}

%\comment{Was sind die grundlegenden Anforderungen an die Benutzungsoberfläche (Bildschirmlayout, Dialogstruktur, ...)?}

 \newpage
  % !TEX encoding = UTF-8 Unicode
% !TEX root =  Pflichtenheft.tex

\section{Qualittszielbestimmungen}

%\comment{Auf welche Qualittsanforderungen (Zuverlssigkeit, Robustheit, Benutzungsfreundlichkeit, Effizienz, ...) wird besonderen Wert gelegt?}

\begin{center}
	\begin{tabular}{l|c|c|c|c}
	                & Sehr Gut & Gut & Normal & Nicht Relevant \\ \hline \hline
	Funktionalität  &          &  x  &        &                \\ \hline
	Zuverlssigkeit  &          &  x  &        &                \\ \hline
	Benutzbarkeit   &          &  x  &        &                \\ \hline
	Effizienz       &          &     &   x    &                \\ \hline
	Änderbarkeit    &          &     &        &    x           \\ \hline
	Übertragbarkeit &          &     &        &    x           \\
	\end{tabular}
\end{center}
 \newpage
  % !TEX encoding = UTF-8 Unicode
% !TEX root =  Pflichtenheft.tex

\section{Globale Testszenarien und Testfälle}

%\comment{Was sind typische Szenarien, die das Produkt erfllen muss?}

\begin{description}
	\item[/T010/] Das erstellen einer Schaltung mit 5 Eingängen, 5 Ausgängen und einer 50 Gatter wird getestet
	\item[/T020/] Die Ansicht Der Signale für Ausgewählte Leitungen als Grafik wird getestet
	\item[/T030/] Das Speichern und Laden von Schaltungen wird getestet
	\item[/T040/] Das Speichern und Laden von Einganggrößentabellen wird getestet
\end{description} \newpage
  % !TEX encoding = UTF-8 Unicode
% !TEX root =  Pflichtenheft.tex

\section{Entwicklungsumgebung}

%\comment{Welche Software, Hardware und Orgware wird zur Entwicklung bentigt?}


\subsection{Software}

\begin{itemize}
	\item Windows 7 Professional
	\item Visual Studio 2008
	\item Microsoft Visio 2010 Professional
	\item UML2-Designer
	\item Visual C++/CLI und .NET FCL
\end{itemize}

\subsection{Hardware}

\begin{itemize}
	\item Computer der fähig ist Windows mit .NET auszuführen
	\item Tastatur, Monitor, Maus
\end{itemize}

\subsection{Orgware}

\begin{itemize}
	\item keine
\end{itemize}
 \newpage
  % !TEX encoding = UTF-8 Unicode
% !TEX root =  Pflichtenheft.tex

\section{Ergänzungen}

%\comment{Spezielle, noch nicht abgedeckte Anforderungen.}

\subsection{Aufteilung der Aufgaben}

\begin{description}
	\item[Robert Schneider] ~ \\ Der Kern, also die Logik wie die verbunden Gatter miteinander agieren, Kapseln der Gatter als Klassen etc.
	\item[Danny Arnold] ~ \\ Die GUI und alles grafische
	\item[Beide] ~ \\ Die Kommunikation zwischen dem grafischen Teil des Programms und dem Kern über eine API
\end{description} \newpage
  \input{PH12-Glossar}
  \printglossaries
  
\end{document}
