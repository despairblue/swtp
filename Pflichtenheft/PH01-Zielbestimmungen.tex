% !TEX encoding = UTF-8 Unicode
% !TEX root =  Pflichtenheft.tex

\section{Zielbestimmungen}

%\comment{Welche Musskriterien, Wunschkriterien, Abgrenzungskriterien sind erforderlich?}

Es ist ein Programm gefordert, welches es einer Person ermöglicht Boolesche Schaltungen grafisch zu erstellen und zu simulieren.

\subsection{Musskriterien}

\begin{itemize}
	\item Die \gls{GUI}
	
	\begin{itemize}
		\item Kombinatorische Schaltungen erstellen können
		\item Darstellung eines \gls{Canvas} auf dem Gatter per \gls{Drag and Drop} angeordnet werden können
		\item Die Gatter sollen mit Leitungen verbunden werden können
		\item Eingangs- und Ausgangsgrößen sollen als Gatter auf dem gls{Canvas} angeordnet werden können.
		\item Leitungen sollen sich auch Aufspalten können, die Verbindung kann als Gatter realisiert werden.
		

	\end{itemize}

	\item Simulation
	
	\begin{itemize}
		\item Für die Eingänge soll es möglich sein eine Liste von Kombination von Eingangsgrößen angeben zu können (Einganggrößentabelle)
		\item Es soll möglich sein die Eingangsgrößentabelle Kombination für Kombination durchzugehen
		\item Es soll möglich sein die Eingangsgrößentabelle auf einmal abarbeiten zu lassen und sich für jede Zeile die Ausgangsgrößen in einer Tabelle in Verbindung mit der jeweiligen Kombination der Eingangsgrößen ansehen zu können
		\item Ob an einer Leitung ein Strom anliegt oder nicht soll farblich gekennzeichnet
		\item Der Signalverlauf ausgewählter Leitungen soll als Grafik angezeigt werden können
	\end{itemize}

	\item Sonstiges
	
	\begin{itemize}
		\item Das Speichern und Laden von Schaltungen soll möglich sein
		\item Die Eingangsgrößentabelle soll separat von der Schaltung abgespeichert und geladen werden können
	\end{itemize}
	
\end{itemize}

\subsection{Wunschkriterien}

\begin{itemize}
	\item Möglichkeit Sequentielle Schaltungen erstellen und simulieren zu können
\end{itemize}

\subsection{Abgrenzungskriterien}

\begin{itemize}
	\item keine
\end{itemize}
