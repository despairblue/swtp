% !TEX encoding = UTF-8 Unicode
% !TEX root =  Pflichtenheft.tex

\renewcommand{\familydefault}{cmss}

\definecolor{fgcgray}{rgb}{0.4, 0.4, 0.4}
\definecolor{warning}{rgb}{0.9, 0.1, 0.0}
\definecolor{bgctitle}{rgb}{0.5, 0.5, 0.5}
\definecolor{fgctitle}{rgb}{0.95, 0.95, 0.95}

\definecolor{Gray}{gray}{0.8}
\definecolor{lightGray}{gray}{0.925}

\newcolumntype{A}{>{\columncolor{Gray}}c}
\newcolumntype{B}{>{\columncolor{Gray}}l}

\newcommand{\titlefont}[1]{\textcolor{fgctitle}{\fontfamily{cmss}\fontseries{bx}\fontshape{n}\fontsize{20.48}{0pt} \selectfont #1}}
\newcommand{\inversetitlefont}[1]{\textcolor{bgctitle}{\fontfamily{cmss}\fontseries{bx}\fontshape{n}\fontsize{20.48}{0pt} \selectfont #1}}

\addtolength{\oddsidemargin}{-1.0cm}
\addtolength{\evensidemargin}{-1.0cm}
\addtolength{\headwidth}{2.0cm}
\addtolength{\textwidth}{2.0cm}

\setlength{\parindent}{0cm}

\renewcommand{\labelitemi}{$\circ$}
\renewcommand{\labelitemii}{$\diamond$}

\newcommand{\spaceline}[1][8pt]{\vskip #1}
\newcommand{\comment}[1]{\spaceline[5pt] \textcolor{fgcgray}{\scriptsize #1} \spaceline[15pt]}
\newcommand{\attrname}[1]{\textcolor{fgcgray}{\scriptsize #1}}


\makeatletter

\newcommand*{\project}[1]{\gdef\@project{#1}}
\newcommand*{\projectnumber}[1]{\gdef\@projectnumber{#1}}

\newcommand*{\version}[1]{\gdef\@version{#1}}
\newcommand*{\home}[1]{\gdef\@home{#1}}
\newcommand*{\homeref}[1]{\gdef\@homeref{#1}}
\newcommand*{\prerequisite}[1]{\gdef\@prerequisite{#1}}
\newcommand*{\prerequisiteref}[1]{\gdef\@prerequisiteref{#1}}

\newcommand*{\firststudent}[4]{
	\gdef\@firststudentname{#1}
	\gdef\@firststudentmatrnr{#2}
	\gdef\@firststudentstudypath{#3}
	\gdef\@firststudentemail{#4}
}

\newcommand*{\secondstudent}[4]{
	\gdef\@secondstudentname{#1}
	\gdef\@secondstudentmatrnr{#2}
	\gdef\@secondstudentstudypath{#3}
	\gdef\@secondstudentemail{#4}
}

\def\@maketitle{
  %\begin{titlepage}
  \begin{center}
    \colorbox{bgctitle}{
      \parbox{\textwidth}{
        \spaceline
        \centering{\titlefont{\@title}}
        \par
        \spaceline
      }
    }
    \colorbox{white}{
      \parbox{\textwidth}{
        \spaceline
        \centering{\inversetitlefont{\@project}}
        \par
        \spaceline
      }
    }
  \end{center}
  \spaceline[1.5em] {
    \begin{flushright}
    \begin{tabular}[t]{rl}
      \attrname{Projekt:} & \@project ~ \@version \\
      \attrname{Voraussetzung:} & \href{\@prerequisiteref}{\@prerequisite} \\
      \attrname{Autor:} & \@author \\
      \attrname{Home:} & \href{\@homeref}{\@home} \\
      \attrname{letzte "Anderung:} & \@date
    \end{tabular}
    \end{flushright}
    \par
  }
  \spaceline[5.5em]
  %\end{titlepage}
}

\newcommand*{\makesteinbach} {
	\large{
		\begin{center}
			\begin{tabular}{|A|}
				\hline
				\\
				\textbf{\Huge{\quad \quad \quad \quad \quad Pflichtenheft \quad \quad \quad \quad}}
				\\
				\\
				zum Softwareprojekt\\
				(Prof. Steinbach)\\
				\\
				\hline
			\end{tabular}
		\end{center}
	
		\vspace{1mm}
	
		\begin{center}
			\begin{tabular}{|A|}
				\hline
				\\
				\textbf{\quad \quad \quad \@project ~ (\@projectnumber) \quad \quad \quad}
				\\
				\\
				\hline
			\end{tabular}
		\end{center}
		
		\vspace{15mm}
		
		\textbf{Angaben zu den am Projekt beteiligten Studenten:}
		
		\begin{center}
			\normalsize{
				\begin{tabular}{|c|l|l|l|l|}
					\hline
					\rowcolor{Gray} & \textbf{Name, Vorname} & \textbf{Mat.-Nr.} & \textbf{Studiengang} & \textbf{Email-Adresse} \\
					\hline
					\rowcolor{lightGray} \textbf{1.} &  \@firststudentname &  \@firststudentmatrnr &  \@firststudentstudypath &  \@firststudentemail \\
					\hline
					\rowcolor{Gray} \textbf{2}. &  \@secondstudentname&  \@secondstudentmatrnr &  \@secondstudentstudypath &  \@secondstudentemail \\
					\hline
					\rowcolor{lightGray} \textbf{3.} & & & & \\
					\hline
					\rowcolor{Gray} \textbf{4.} & & & & \\
					\hline
					\rowcolor{lightGray} \textbf{5.} & & & & \\
					\hline
				\end{tabular}
			}
		\end{center}
		
		\vfill
		
		\begin{center}
			\begin{tabular}{|BB|}
				\hline
				\textbf{Bestätigt durch Prof. Steinbach} & \quad \quad \quad \quad \quad \quad \quad \quad \quad \\
				\textbf{Datum, Unterschrift} &
				\\
				\hline
			\end{tabular}
		\end{center}
	}
}


\setcounter{secnumdepth}{4}
\setcounter{tocdepth}{4}
	 
\newcounter{subsubsubsection}[subsubsection]
\def\subsubsubsectionmark#1{}
\def\thesubsubsubsection{\thesubsubsection .\arabic{subsubsubsection}}
\def\subsubsubsection{\@startsection{subsubsubsection}{4}{\z@}{-3.25ex plus -1 ex minus -.2ex}{1.5ex plus .2ex}{\normalsize\bf}}
\def\l@subsubsubsection{\@dottedtocline{4}{4.8em}{4.2em}}

\makeatother

\everytexdraw{
  \drawdim cm \linewd 0.01
  \arrowheadtype t:T
  \arrowheadsize l:0.2 w:0.2
  \setgray 0.5
}

\newcommand{\xheight}{0.6}
\newcommand{\xlength}{0.6}
\newcommand{\yheighta}{1.0}
\newcommand{\yheightb}{0.8}
\newcommand{\yheightc}{0.6}
\newcommand{\yheightd}{0.5}
\newcommand{\yheighte}{0.4}
\newcommand{\yheightf}{0.35}

\newcommand{\xhline}{\rlvec({\xlength} 0)}
\newcommand{\xharrow}{\ravec(0.7 0)}
\newcommand{\xnext}{%\rlvec(0.05 0) \lpatt(0.04 0.04) \rlvec(0.15 0) \lpatt()
}

\newcommand{\xtext}[3][\xheight]{
  \bsegment
    \bsegment
      \setsegscale 0.5
      \textref h:L v:C  \htext({\xheight} -0.1){#3}
    \esegment
    \setsegscale 0.5 \lvec(0 #1)
    \setsegscale 1
    \rlvec(#2 0) \rlvec(0 -#1) \rlvec(-#2 0) \lvec(0 0)
    \savepos(#2 0)(*@x *@y)
  \esegment
  \move(*@x *@y)
}

\newcommand{\bxtext}[3][\xheight]{
  \setgray{0.1}
  \linewd{0.026}
  \xtext[\xheight]{#2}{#3}
  \linewd{0.01}
  \setgray{0.5}
}

\newcommand{\xstartpage}{\bxtext{2.1}{Startseite}}
\newcommand{\xmainpage}{\bxtext{2.3}{Hauptseite}}
\newcommand{\xusermenu}{\bxtext{2.9}{Benutzermenu}}
\newcommand{\xgamelist}{\bxtext{2.2}{Spieleliste}}
\newcommand{\xportfolio}{\bxtext{2.0}{Portfolio}}
\newcommand{\xaccount}{\xtext{3.8}{Kennung per \textsl{eMail}}}
