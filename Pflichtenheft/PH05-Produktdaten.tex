% !TEX encoding = UTF-8 Unicode
% !TEX root =  Pflichtenheft.tex

\section{Produktdaten}

%\comment{Was speichert das Produkt (langfristig) aus Benutzersicht?}

Es sollen folgende Daten persistent gespeichert werden. \\
Die Speicherung erfolgt in einer simplen Textdatei (\gls{CSV}).\\
Eingangsgrößentabellen werden in separaten Texdateien gespeichert.
\begin{description}
	\item[D010/] \emph{Gatterdaten:} Alle Informationen zu einem Gatter.
	\begin{itemize}
		\item \emph{Typ}
		\item \emph{Name} 
		\item \emph{Position} 
	\end{itemize}
	
	\item[/D020/] \emph{Leitungsdaten:} Alle Informationen zu einer Leitung
	\begin{itemize}
		\item \emph{Name}
		\item \emph{Gatter A:} Das Gatter mit dessen Eingang die Leitung verbunden wird
		\item \emph{Gatter B:} Dast Gatter mit dessem Ausgang die Leitung verbunden wird
	\end{itemize}
	
	\item[/D030/] \emph{Eingangsgrößendaten:} Alle Information zu den Eingangsgrößen
	\begin{itemize}
		\item \emph{Name}
		\item \emph{Position}
		\item \emph{1 oder 0}
	\end{itemize}
	
	\item[/D040/] \emph{Ausgangsgrößen:} Alle Informationen zu den Ausgangsgrößen
	\begin{itemize}
		\item \emph{Name}
		\item \emph{Position}
	\end{itemize}
	
	\item[/D050/] \emph{Eingangsgrößentabelle:} Alle Informationen zu einer Eingangsgrößentabelle
	\begin{itemize}
		\item \emph{Anzahl der Eingänge}
		\item \emph{Die verschiedenen Belegungen der Eingänge}
	\end{itemize}
\end{description}