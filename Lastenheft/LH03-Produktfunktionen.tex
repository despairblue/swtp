% !TEX root = Lastenheft.tex
% !TEX encoding = UTF-8 Unicode

\section{Produktfunktionen}

\comment{Was sind die Hauptfunktionen des Produktes aus der Sicht des Auftraggebers?}

\subsection{Schaltplanerstellung}

Der Benutzer soll die Möglichkeit haben einen Schaltplan zu erstellen.
\begin{description}
	\item[/LF110/] Möglichkeit Gatter von einer Toolbar per Drag and Drop in einem bestimmten Bereich des Fensters anzuordnen.
	\item[/LF120/] Möglichkeit die Gatter mittels Leitungen miteinander zu verbinden.
	\item[/LF130/] Möglichkeit Eingangsgrößen festzulegen.
\end{description}

\subsection{Simulation}

Der Benutzer soll den von ihm erstellten Schaltplan simulieren können. 
\begin{description}
	\item[/LF210/] Möglichkeit Takt für Takt durchzugehen
	\begin{description}
		\item[/LF220/] Die anliegenden Werte an den jeweiligen Gattern sollen dabei visualisiert werden.
	\end{description}
	\item[/LF230/] Möglichkeit alle Zwischenergebnisse nach x Takten tabellarisch sich auszugeben lassen
\end{description}

\subsection{Sonstiges}
Der Benutzer kann Schaltungen speichern und laden.
\begin{description}
	\item[/LF310/] Die aktuelle Schaltung kann in einer Datei gespeichert werden.
	\item[/LF320/] Früher gespeicherte Schaltungen können wieder geladen werden.
\end{description}