% !TEX root = Lastenheft.tex
% !TEX TS-program = pdflatex
% !TEX encoding = UTF-8 Unicode

\definecolor{fgcgray}{rgb}{0.4, 0.4, 0.4}

\definecolor{bgctitle}{rgb}{0.0, 0.0, 0.5} 
\definecolor{fgctitle}{rgb}{0.99, 0.99, 0.95} 
\definecolor{Gray}{gray}{0.8}
\definecolor{lightGray}{gray}{0.925}

\newcolumntype{A}{>{\columncolor{Gray}}c}
\newcolumntype{B}{>{\columncolor{Gray}}l}

\newcommand{\titlefont}[1] {
	\textcolor{fgctitle} {
		\fontfamily{cmss}
		\fontseries{bx}
		\fontshape{n}
		\fontsize{20.48}{0pt} \selectfont #1
	}
}
 
\newcommand{\inversetitlefont}[1]{
	\textcolor{bgctitle} {
		\fontfamily{cmss}
		\fontseries{bx}
		\fontshape{n}
		\fontsize{20.48}{0pt} \selectfont #1
	}
}

\setlength{\headheight}{13pt}

\addtolength{\oddsidemargin}{-1.0cm}
\addtolength{\evensidemargin}{-1.0cm} 
\addtolength{\headwidth}{2.0cm} 
\addtolength{\textwidth}{2.0cm}

\setlength{\parindent}{0cm}

\renewcommand{\labelitemi}{$\circ$}

\newcommand{\comment}[1]{\vskip 5pt \textcolor{fgcgray}{\scriptsize #1} \vskip 15pt} 
\newcommand{\attrname}[1]{\textcolor{fgcgray}{\scriptsize #1}}

\makeatletter

\newcommand*{\project}[1]{\gdef\@project{#1}} 
\newcommand*{\projectnumber}[1]{\gdef\@projectnumber{#1}}

\newcommand*{\firststudent}[4]{
	\gdef\@firststudentname{#1}
	\gdef\@firststudentmatrnr{#2}
	\gdef\@firststudentstudypath{#3}
	\gdef\@firststudentemail{#4}
}

\newcommand*{\secondstudent}[4]{
	\gdef\@secondstudentname{#1}
	\gdef\@secondstudentmatrnr{#2}
	\gdef\@secondstudentstudypath{#3}
	\gdef\@secondstudentemail{#4}
}

\def\@maketitle { 
	\begin{center}
	
		\colorbox{bgctitle} {
			\parbox{\textwidth} {
				\vskip 8pt 
				\centering {
					\titlefont{\@title}
				}
				\vskip 8pt 
			} 
		}
		
		\colorbox{white} {
			\parbox{\textwidth} {
				\vskip 8pt
				\centering{\inversetitlefont{\@project}}
				\vskip 8pt
			} 
		}
		
	\end{center}
	
	\vskip 1.0em {
		\lineskip .5em 
		\begin{flushright}
			\begin{tabular}[t]{rl}
				\attrname{Projekt:} & \@project ~ \@version \\
				\attrname{Autor:} & \@author \\
				\attrname{Home:} & \href{\@homeref}{\@home} \\
				\attrname{letzte Änderung:} & \@date 
			\end{tabular}
		\end{flushright}
		\par 
	}
	
	\vskip 5.5em
}

\newcommand*{\makesteinbach} {
\large{
	\begin{center}
		\begin{tabular}{|A|}
			\hline
			\\
			\textbf{\Huge{\quad \quad \quad \quad \quad Lastenheft    \quad \quad \quad \quad}}
			\\
			\\
			zum Softwareprojekt\\
			(Prof. Steinbach)\\
			\\
			\hline
		\end{tabular}
	\end{center}

	\vspace{1mm}

	\begin{center}
		\begin{tabular}{|A|}
			\hline
			\\
			\textbf{\quad \quad \quad \@project ~ (\@projectnumber) \quad \quad \quad}
			\\
			\\
			\hline
		\end{tabular}
	\end{center}
	
	\vspace{15mm}
	
	\textbf{Angaben zu den am Projekt beteiligten Studenten:}
	
	\begin{center}
		\normalsize{
			\begin{tabular}{|c|l|l|l|l|}
				\hline
				\rowcolor{Gray} & \textbf{Name, Vorname} & \textbf{Mat.-Nr.} & \textbf{Studiengang} & \textbf{Email-Adresse} \\
				\hline
				\rowcolor{lightGray} \textbf{1.} &  \@firststudentname &  \@firststudentmatrnr &  \@firststudentstudypath &  \@firststudentemail \\
				\hline
				\rowcolor{Gray} \textbf{2}. &  \@secondstudentname&  \@secondstudentmatrnr &  \@secondstudentstudypath &  \@secondstudentemail \\
				\hline
				\rowcolor{lightGray} \textbf{3.} & & & & \\
				\hline
				\rowcolor{Gray} \textbf{4.} & & & & \\
				\hline
				\rowcolor{lightGray} \textbf{5.} & & & & \\
				\hline
			\end{tabular}
		}
	\end{center}
	
	\vfill
	
	\begin{center}
		\begin{tabular}{|BB|}
			\hline
			\textbf{Bestätigt durch Prof. Steinbach} & \quad \quad \quad \quad \quad \quad \quad \quad \quad \\
			\textbf{Datum, Unterschrift} &
			\\
			\hline
		\end{tabular}
	\end{center}
}
}

\setcounter{secnumdepth}{4}
\setcounter{tocdepth}{4}

\newcounter{subsubsubsection}[subsubsection] 
\def\subsubsubsectionmark#1{} 
\def\thesubsubsubsection{\thesubsubsection .\arabic{subsubsubsection}} 
\def\subsubsubsection{\@startsection{subsubsubsection}{4}{\z@}{-3.25ex plus -1 ex minus -.2ex}{1.5ex plus .2ex}{\normalsize\bf}} 
\def\l@subsubsubsection{\@dottedtocline{4}{4.8em}{4.2em}}

\makeatother 
