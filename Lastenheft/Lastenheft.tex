% !TEX TS-program = Make
% !TEX encoding = UTF-8 Unicode

\documentclass[a4paper,10pt]{article} 
\usepackage[utf8]{inputenc} 
\usepackage[ngerman]{babel} 
\usepackage{coordsys,logsys,color} 
\usepackage{fancyhdr} 
\usepackage{hyperref}
%\usepackage{color}
\usepackage{colortbl}
\usepackage[xindy, acronym, toc, style=altlist]{glossaries}

\definecolor{darkblue}{rgb}{0,0,.6} 

\hypersetup{colorlinks=true, breaklinks=true, linkcolor=darkblue, menucolor=darkblue, urlcolor=darkblue, citecolor=darkblue}

\pagestyle{fancy}

\makeglossary
% !TEX encoding = UTF-8 Unicode
% !TEX root = Pflichtenheft.tex

\newacronym{GUI}{GUI}{Graphical User Interface}

\newglossaryentry{Canvas} {
	name={Canvas},
	description={}
}

\newglossaryentry{Drag and Drop} {
	name={Drag and Drop},
	description={}
}

% !TEX root = Lastenheft.tex
% !TEX TS-program = pdflatex
% !TEX encoding = UTF-8 Unicode

\definecolor{fgcgray}{rgb}{0.4, 0.4, 0.4}

\definecolor{bgctitle}{rgb}{0.0, 0.0, 0.5} 
\definecolor{fgctitle}{rgb}{0.99, 0.99, 0.95} 

\newcommand{\titlefont}[1] {
	\textcolor{fgctitle} {
		\fontfamily{cmss}
		\fontseries{bx}
		\fontshape{n}
		\fontsize{20.48}{0pt} \selectfont #1
	}
}
 
\newcommand{\inversetitlefont}[1]{
	\textcolor{bgctitle} {
		\fontfamily{cmss}
		\fontseries{bx}
		\fontshape{n}
		\fontsize{20.48}{0pt} \selectfont #1
	}
}

\setlength{\headheight}{13pt}

\addtolength{\oddsidemargin}{-1.0cm}
\addtolength{\evensidemargin}{-1.0cm} 
\addtolength{\headwidth}{2.0cm} 
\addtolength{\textwidth}{2.0cm}

\setlength{\parindent}{0cm}

\renewcommand{\labelitemi}{$\circ$}

\newcommand{\comment}[1]{\vskip 5pt \textcolor{fgcgray}{\scriptsize #1} \vskip 15pt} 
\newcommand{\attrname}[1]{\textcolor{fgcgray}{\scriptsize #1}}

\makeatletter

\newcommand*{\project}[1]{\gdef\@project{#1}} 
\newcommand*{\version}[1]{\gdef\@version{#1}} 
\newcommand*{\home}[1]{\gdef\@home{#1}} 
\newcommand*{\homeref}[1]{\gdef\@homeref{#1}}

\def\@maketitle { 
	\begin{center}
	
		\colorbox{bgctitle} {
			\parbox{\textwidth} {
				\vskip 8pt 
				\centering {
					\titlefont{\@title}
				}
				\vskip 8pt 
			} 
		}
		
		\colorbox{white} {
			\parbox{\textwidth} {
				\vskip 8pt
				\centering{\inversetitlefont{\@project}}
				\vskip 8pt
			} 
		}
		
	\end{center}
	
	\vskip 1.0em {
		\lineskip .5em 
		\begin{flushright}
			\begin{tabular}[t]{rl}
				\attrname{Projekt:} & \@project ~ \@version \\
				\attrname{Autor:} & \@author \\
				\attrname{Home:} & \href{\@homeref}{\@home} \\
				\attrname{letzte Änderung:} & \@date 
			\end{tabular}
		\end{flushright}
		\par 
	}
	
	\vskip 5.5em
}

\setcounter{secnumdepth}{4}
\setcounter{tocdepth}{4}

\newcounter{subsubsubsection}[subsubsection] 
\def\subsubsubsectionmark#1{} 
\def\thesubsubsubsection{\thesubsubsection .\arabic{subsubsubsection}} 
\def\subsubsubsection{\@startsection{subsubsubsection}{4}{\z@}{-3.25ex plus -1 ex minus -.2ex}{1.5ex plus .2ex}{\normalsize\bf}} 
\def\l@subsubsubsection{\@dottedtocline{4}{4.8em}{4.2em}}

\makeatother 


\begin{document} 
	% !TEX root = Lastenheft.tex
% !TEX encoding = UTF-8 Unicode

\lhead{\sc{Lastenheft Lgk Smltr}}


\project{Logik Simulator}
\projectnumber{30} 
\firststudent{Robert Schneider}{52588}{BAI}{rob.schneider@student.tu-freiberg.de}
\secondstudent{Danny Arnold}{52315}{BAI}{danny.arnold@student.tu-freiberg.de}


\makesteinbach

\thispagestyle{empty}

\newpage

\tableofcontents 
 
	\newpage
	% !TEX root = Lastenheft.tex
% !TEX encoding = UTF-8 Unicode
\section{Zielbestimmungen}

\comment{Welche Ziele sollen durch den Einsatz der Software erreicht werden?}

Dem Benutzer soll die Möglichkeit gegeben werden eine Schaltung zu entwerfen und ihr Verhalten zu simulieren. 
	\newpage
	% !TEX root = Lastenheft.tex
% !TEX encoding = UTF-8 Unicode

\section{Produkteinsatz}

\comment{Für welche Anwendungsbereiche und Zielgruppen ist die Software vorgesehen?}

Es kann zum Erwerb von Grundlagen bezüglich technischer Komponenten und ihrem Zusammenwirken genutzt werden. Zielgruppe sind sowohl Privatpersonen als auch Studenten. 
	\newpage
	% !TEX root = Lastenheft.tex
% !TEX encoding = UTF-8 Unicode

\section{Produktfunktionen}

\comment{Was sind die Hauptfunktionen des Produktes aus der Sicht des Auftraggebers?}

\subsection{Schaltplanerstellung}

Der Benutzer soll die Möglichkeit haben einen Schaltplan zu erstellen.
\begin{description}
	\item[/LF110/] Möglichkeit Gatter von einer Toolbar per Drag and Drop in einem bestimmten Bereich des Fensters anzuordnen.
	\item[/LF120/] Möglichkeit die Gatter mittels Leitungen miteinander zu verbinden.
	\item[/LF130/] Möglichkeit Eingangsgrößen festzulegen.
\end{description}

\subsection{Simulation}

Der Benutzer soll den von ihm erstellten Schaltplan simulieren können. 
\begin{description}
	\item[/LF210/] Möglichkeit Takt für Takt durchzugehen
	\begin{description}
		\item[/LF220/] Die anliegenden Werte an den jeweiligen Gattern sollen dabei visualisiert werden.
	\end{description}
	\item[/LF230/] Möglichkeit alle Zwischenergebnisse nach x Takten tabellarisch sich auszugeben lassen
\end{description}

\subsection{Sonstiges}
Der Benutzer kann Schaltungen speichern und laden.
\begin{description}
	\item[/LF310/] Die aktuelle Schaltung kann in einer Datei gespeichert werden.
	\item[/LF320/] Früher gespeicherte Schaltungen können wieder geladen werden.
\end{description} 
	\newpage
	% !TEX root = Lastenheft.tex
% !TEX encoding = UTF-8 Unicode

\section{Produktdaten}

\comment{Was sind die Hauptdaten des Produktes aus der Sicht des Auftraggebers?}

Es sollen folgende Daten persistent gespeichert werden. \\
Die Speicherung erfolgt in einer simplen Textdatei (\gls{CSV})
\begin{description}
	\item[/LD010/] \textit{Gatterdaten:} Alle Informationen zu einem Gatter.
	\begin{itemize}
		\item Name 
		\item Position 
		\item Gatter welche an dem Eingang oder den Eingängen anliegen 
	\end{itemize}
	
	\item[/LD020/] \textit{Eingangsgrößendaten:} Alle Information zu den Eingangsgrößen
	\begin{itemize}
		\item Name
		\item Position
		\item 1 oder 0
	\end{itemize}
\end{description} 
	\newpage
	% !TEX root = Lastenheft.tex
% !TEX encoding = UTF-8 Unicode

\section{Produktleistungen}

%\comment{Werden für bestimmte Funktionen besondere Ansprüche in Bezug auf Zeit, Datenumfang oder Genauigkeit gestellt?}


Es werden keine besonderen Anforderungen gestellt. 
	\newpage
	% !TEX root = Lastenheft.tex
% !TEX encoding = UTF-8 Unicode
\section{Qualitätsanforderungen}

%\comment{Auf welche Qualitätsanforderungen (Zuverlässigkeit, Robustheit, Benutzungsfreundlichkeit, Effizienz, ...) wird besonderen Wert gelegt?}

\begin{tabular}{|c|c|c|c|c|}
	\hline Produktqualität & Sehr gut & Gut & Normal & Nicht relevant \\
	\hline Funktionalität  &          &     &   x    &                \\
	\hline Zuverlässigkeit &    x     &     &        &                \\
	\hline Benutzbarkeit   &          &     &   x    &                \\
	\hline Effizienz       &          &     &        &       x        \\
	\hline Änderbarkeit    &          &     &   x    &                \\
	\hline Übertragbarkeit &          &     &        &       x        \\
	\hline 
\end{tabular} 
	\newpage
	% !TEX root = Lastenheft.tex
% !TEX encoding = UTF-8 Unicode

\section{Ergänzungen}

%\comment{Gibt es noch aussergewöhnliche Anforderungen?}

\subsection{Realisierung}

Das vorliegende System muss mit \textit{Microsoft Visual C++/CLI} und \textit{FCL} realisiert werden. 
	\newpage
	\printglossaries
\end{document} 
